\chapter{Introduction}

\section{Background}

It is generally useful for software to be able to programmatically determine the
capabilities of the hardware it is running on. In addition external debuggers
need to know the same information. Instead of having each RISC-V extension
specify their own discovery mechanism, this document attempts to specify a
single format and access method that can be used to describe any feature
software and debuggers might want to be aware of.

\section{Goals}

\begin{steps}{The task group will define:}
\item A machine-readable format for system descriptions. It's intended to be
present in hardware and accessible at all times, so software and debuggers
can read this description.
\item A mechanism for software and debuggers to access the system description,
if it exists.
\item A human-readable format for system descriptions. It's intended to be used
to talk about the system description in this document as well as other
documentation, to help designers write a system description, and for debuggers
to display the system description to their users.
\item Software that translates back and forth between the machine-readable and
human-readable format.
\item The equivalent syntax and format of system description in external
industrial standards (e.g. Devicetree, SMBIO, ACPI). It's intended to be used
by software that translates system description into those formats as part of
the system boot process.
\end{steps}

\section{Use Cases}

\subsection{External Debuggers}

When an external debugger connects to a system, it would like to discover as
much as possible about that system in as little time as possible. Some of this
is merely to show the user (e.g. a manufacturer name), while other features are
critical to the user (e.g. XLEN), and other features determine what kind of
operations the user can perform (e.g. supported hardware trigger types). Most
of these are already discoverable, although many require writing a value and
checking the result to see whether support exists.

In the case of hardware trigger types there are so many possible combinations
of options that this is prohibitive. In other cases the software to detect all
features is quite complex. When some harts are powered down, no discovery is
possible on them at all. Supplying a system description can solve all these
problems.

\subsection{System Firmware}

Typical system firmware is executed when the system is powered on. It
initializes hardware, and builds up firmware services or data structures for
booting up system to OS. Examples are uboot for embedded systems, and BIOS for
PCs.

By accessing the system description (Section~\ref{sec:AccessMethod}), firmware
can programmatically determine the hardware capabilities and configure hardware
accordingly.  These hardware capabilities can include availability and
implemented features of Physical Memory Protection (PMP), Core Local Interrupt
Controller (CLIC), Core Local Interruptor (CLINT), memory map, Platform-Level
Interrupt Controller (PLIC), Real Time Clock (RTC), reset mechanism, and any
future optional core features.

A system description is an efficient alternative to testing for specific
hardware feature (including handling failures) or customizing system firmware
for the specific system it will run on.

Often system firmware will convert the system description into another
representation which is defined in the external industrial standards
(Appendix~\ref{sec:ExternalIndustrialStandard}) and passed to the subsequent
boot processes.

\begin{commentary}
The system firmware stated herein refer to the startup software which is executed
when system power on. The system firmware initializes hardware configuration and
builds up firmware services or data structures for booting up system to OS. The typical
system firmware such as uboot for embedded systems, BIOS for PCs or other firmware
frameworks.
\end{commentary}

\chapter{Machine-Readable Format}

\begin{steps}{Ideas:}
\item Fully custom, along the lines of
    \url{https://github.com/riscv/riscv-debug-spec/pull/450/files#diff-9cee87c1f0feb10d9e3e9b2bfad92985R380}
\item Semi-custom, using CBOR
\item ...
\end{steps}

\chapter{Human-Readable Format}

\begin{steps}{Ideas:}
\item Take the binary format, and expand constants to strings, add whitespace,
keeping the mapping as close to 1:1 as possible.
\item XML
\item JSON
\item ...
\end{steps}

\chapter{Access Method}
\label{sec:AccessMethod}

\begin{steps}{Ideas:}
\item A CSR contains the base address of where the structure can be accessed on
the system bus
\item A CSR address/data pair can be used to access the structure. (Write index
to the address CSR, then read from the data CSR.)
\item ...
\end{steps}
